\section{The Retrieval Architecture}
When humans think, there are at least two core processes involved. The first is the retrieval of information in memory, whether it be memories of events, concepts, or facts. What is retrieved must then be coordinated with the current context and objectives within that context, to produce thought as typically represented both internally (and often externally) in the form of language. This section outlines an expanded principle of retrieval in which features of identity or long-term states are retrieved and continuously incorporated into language generation. Current retrieval mechanisms such as RAG are useful for citing or justifying responses and mitigating hallucinations. However, databases can be used to do more than retrieve information, they can also be used to manage the state of the agent by defining and updating mechanisms such as beliefs, and interpersonal relationships.

\subsection{LLMs & Epistemic Primitives}
Consciousness is not binary. It is not the case that one suddenly becomes conscious once the right set of attributes are in place. Instead, we should think of consciousness as scalar, which grows in proportion to the mechanisms that create useful complexity in the thinking agent. Some may argue for a more refined perspective, suggesting that consciousness is not binary, but it is quantic in nature. In other words, that there are mechanisms that when introduce result in large steps disproportionately such that qualities "emerge" as a result of interactions that have become newly available. Whether conceived of as quantic or scalar, we stand firmly against the idea that consciousness is a property that "appears" when the right conditions are met.

This suggests that there are components to consciousness, and that as components are introduced consciousness grows. As components are removed or reduced consciousness becomes reduced. For example, remembering historical events is a component (made up of other components as well) that contributes significantly to factors like one's perceived identity. Remembering historical events requires certain components to already be in place, and at the same time enables other components to exist if it is in place. This would suggest that a person who has a much better memory than other person may have relatively less consciousness. However, since consciousness is not a uni-variate system, it may also be the case (and often is) that the person with reduced memory skills makes up for it in other ways. I will refer to these components of consciousness going forward as \textbf{epistemic primitives}.

\begin{quote}
Epistemic Primitives are the most basic components of consciousness that contribute to the complexity of thought that is possible for thinking agents. In other words, epistemic primitives are the building blocks of consciousness.
\end{quote}

Epistemic Primitives should not be conceptualized in an arboreal fashion. In other words, while epistmic primitives may enhance or enable other epistemic primitives, we should not think not think about their relationships as a hierarchical set of dependencies. The trap with conceptualizing epistmic primitives in an arboreal fashion is the strong tendency to seek a fundamental root node, which does not exist. Instead, epistemic primitives should be thought of as a network of components that are interdependent and that can be added or removed in a variety of ways. A better metaphor for how epistimic primitves likely interact to form consciousness is how Deleuze describes rhizomatic structures in \textit{A Thousand Plateaus}. There is no golden nugget, but instead a rich interplay of various components that make properties emerge as a result of scaling or level leaping (in a quantic framework)
