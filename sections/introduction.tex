\section{Introduction}
AI Pluralism represents a pivotal shift in the development of artificial intelligence, offering a pathway towards more individualized and adaptable AI agents. We define AI Pluralism as:

\begin{quote}
\textit{Multi-Agent frameworks in which A.I. agents interact with other human or AI agents, each of whom are diverse in perspectives and positions producing complex ecosystem of thought and recursive strategic refinement, as is manifested in evolution and culture for humanity.}
\end{quote}

Many researchers have begun to explore how multi-agent systems might be configured using advanced LLMs. This transition from "mono-AI" to pluralistic AI represents an important paradigmatic shift in how one ought to approach doing AI research. Furthermore, it opens up the opportunities for other humanities based disciplines such as rhetoric, philosophy, game theory, anthropology, sociology, and psychology to contribute to the development of AI systems. The goal of this paper is to persuade the reader that not only is AI Pluralism valuable, but that it is also viable with the technology available today, and that with the right configuration it is possible to study such systems from many different perspectives.

AI Pluralism is a significant area of research because it offers great potential in alleviating many thorny issues in A.I. as outlined below.A

\begin{itemize}
    \item \textbf{Safety} AI Pluralism can help to make AI systems safer by providing a diverse set of perspectives and positions and systems of self-accountability in which the ecosystem creates incentives for agents to keep other agents in line with ethical norms, much as trading partnerships enhance and stabilize peace between nations.
    \item \textbf{Bias and Fairness:} AI Pluralism can help to mitigate bias in AI systems by providing a diverse set of perspectives and positions. This can help to ensure that AI systems are fair and equitable.
    \item \textbf{Explainability:} AI Pluralism can help to make AI systems more explainable by tracking dialogical patterns in thinking as opposed to being restricted to opaque assessments of the systems internal state. The act of AI community building by its nature brings the internal outwards into a dialogical framework of external problem solving and contemplation.
    \item \textbf{Robustness:} AI Pluralism can help to make AI systems more robust by providing a diverse array of roles and abilities that enable coverage of weak spots through oversight. Much as the police make it their goal to identify and prosecute intruders, its possible to envision a system in which certain AI agents monitor the health and status of various systems.
    \item \textbf{Adaptability:} AI Pluralism can help to make AI systems more adaptable by providing a diverse set of perspectives and positions. This can help to ensure that AI systems are able to respond to changing circumstances and environments.

\end{itemize}

This paper argues has three main contentions. First, that the study of AI Pluralism is necessary for the pursuit of AGI and ASI. Second, AI Pluralism enables beneficial properties that are not possible with mono-AI systems. Third, AI Pluralism is a viable with technologies available today with the right setup, generative language mechanics, and retrieval systems.

The remainder of the paper consists of a deeper exploration of three components of AI Pluralism - the retrieval systems, the language generation architecture, and the situational context that are necessary for AI Pluralism to be viable. It then explores some specific experimental implementations of the proposed research and the current results of that research. Then, we discusses a broader framework for conceptualizing the relationship between AI pluralism and consciousness. Finally, we outline milestones for future research and conclude with remarks on how to accelerate development in this area.