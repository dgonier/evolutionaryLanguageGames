\section{The Situation}

Language games, as conceived by Ludwig Wittgenstein, describe the various forms of language used in different contexts that follow specific rules. In the context of artificial intelligence, language games can serve as structured frameworks for agents to interact and evolve through language-based tasks. These games are pivotal in advancing language understanding and generation capabilities of AI systems. Wittgenstein pointed out that language is always contextualized, and is ultimately meaningless outside of an applied context. We draw upon this observation, when incorporating game structures into the very dynamics of AI agent training. How do agents coopoerate, compete, and coordinate with one another? How do they use language as internal mechanisms as well as outward responses.

In our particular use case, language games refer to open-ended multi-agent games with winners, losers, and other mechanisms of scoring, that require strategic, concise, and effective communication to be successful. Ideally, agents that are very successful in such games, will also be successful in other similar contexts. For example, an agent(s) with strong debating skills, is also likely to give good balanced advice that represents multiple perspectives.

The use of games and specifically language games is not new to A.I. research. It involves the intersection of language theory, computer science, and cognitive psychology, providing a medium through which agents can learn and reason. Notably, Alan Turing's proposal of the Turing Test as a measure of machine intelligence hinges on the machine's ability to use language indistinguishably from a human, which underscores the centrality of language in AI development. Turing was right to identify the game structure as a powerful medium for evaluating progress in machine intelligence.

\subsection{Examples of Language Games}

Language games in AI can vary widely, each serving different purposes from training models to testing their capabilities:

\begin{itemize}
    \item \textbf{Persuasion Party:} Agents attempt to persuade one another within a set of rules, aiming to win over others to their point of view.
    \item \textbf{Murder Mystery:} Agents work collaboratively to solve a fictional crime, using deductive reasoning and dialogue.
    \item \textbf{Guess Who:} Agents ask questions about a hidden person's beliefs and relationships, and focus on identifying who the hidden entity is.
    \item \textbf{Debate (Policy, Lincoln-Douglas, etc.):} Structured formats where agents argue for or against specific positions, simulating a competitive debate environment.
    \item \textbf{20 Questions:} One agent thinks of an object, and another asks up to 20 questions to guess it, showcasing the agent's ability to ask informative questions and make deductions.
\end{itemize}

\subsection{Incorporating Language Games as a Mechanism for Model Evolution}

Integrating language games into AI development helps in several ways. It provides a controlled yet flexible environment to develop and refine AI capabilities. Through iterative gameplay, AI agents can improve their language understanding, strategic planning, and interaction with humans and other AI agents. These interactions are crucial for the evolution of AI, pushing the boundaries of what artificial agents can achieve in complex social interactions and decision-making scenarios.
