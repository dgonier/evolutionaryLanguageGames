\newcommand{\qTower}{
``
It is an outrageously oversimplified structure, but idealization is the price one should often be willing to pay for synoptic insight. I call it the Tower of Generate-and-Test; as each new floor of the Tower gets constructed, it empowers the organisms at that level to find better and better moves, and find them more efficiently
    ...
%darwinian
A variety of candidate organisms were blindly generated by more or less arbitrary processes of recombination and mutation of genes. These organisms were field-tested, and only the best designs survived. This is the ground floor of the Tower. Let us call its inhabitants Darwinian creatures.
    ...
% skinnerian
These individuals thus confronted the environment by generating a variety of actions, which they tried out, one by one, until they found one that worked. We may call this subset of Darwinian creatures, the creatures with conditionable plasticity, Skinnerian creatures, since, as B. F. Skinner was fond of pointing out, operant conditioning is not just analogous to Darwinian natural selection; it is continuous
with it. “Where inherited behavior leaves off, the inherited modifiability of the process of conditioning takes over”
    ...

% popperian
    Skinnerian conditioning is a fine capacity to have, so long as you are not killed by one of your early errors. A better system involves preselection among all the possible behaviors or actions, weeding out the truly stupid options before risking them in the harsh world. We human beings are creatures capable of this third refinement, but we are not alone. We may call the beneficiaries of this third story in the Tower Popperian creatures, since, as Sir Karl Popper once elegantly put it, this design enhancement “permits our hypotheses to die in our stead.”
...

    The information about the world has to be there, but it also has to be structured in such a way that there is a nonmiraculous explanation of how it got there, how it is maintained, and how it actually achieves the preselective effects that are its raison d’être.

...
    This is where earlier design decisions come back to haunt—to constrain—the designer. In particular, choices that evolution has already made between need-to-know and commando-team now put major constraints on the options for design improvement. If a particular species’ brain design has already gone down
    the need-to-know path with regard to some control problem, only minor modifications (fine tuning, you might say) can be readily made to the existing structures, so the only hope of making a major revision of the internal environment to account for new problems, new features of the external environment that matter, is to submerge the old hard-wiring under a new layer of pre-emptive control.

    % Gregorian
    ...
The imitative actions we share with some higher animals may show the benefits of information gathered not just by our ancestors, but also by our social groups over generations, transmitted nongenetically by a “tradition” of imitation. But our more deliberatively planned acts show the benefits of information gathered and transmitted by our conspecifics in every culture, including, moreover, items of information that no single individual has embodied or understood in any sense. And though some of this information may be of rather ancient acquisition, much of it is brand-new. When comparing the time scales of genetic and cultural evolution, it is useful to bear in mind that we today—every one of us—can easily understand many ideas that were simply unthinkable by the geniuses in our grandparents’ generation! We may call this sub-sub-subset of Darwinian creatures Gregorian creatures, since the British psychologist Richard Gregory is to my mind the pre-eminent theorist of the role of information (or, more exactly, what Gregory calls Potential Intelligence) in the creation of Smart Moves (or what Gregory calls Kinetic Intelligence). Gregory observes that a pair of scissors, as a well-designed artifact, is not just a result of intelligence, but an endower of intelligence (external Potential Intelligence), in a very straightforward and intuitive sense: when you give someone a pair of scissors, you enhance his potential to arrive more safely and swiftly at Smart Moves (Gregory 1981, pp. 311ff).
    ...
Skinnerian creatures ask themselves, “What do I do next?” and haven’t a clue how to answer until they have taken some hard knocks. Popperian creatures make a big advance by asking themselves, “What should I think about next?” before they ask themselves, “What should I do next?” Gregorian creatures take a further big step by learning how to think better about what they should think about next—and so forth, a tower of further internal reflections with no fixed or discernible limit.

    ...

There is one more embodiment of that wonderful idea, and it is the one that gives our minds their greatest power: once we have language—a bountiful kit of mind-tools—we can use these tools in the structure of deliberate, foresightful generate-and-test known as science. All the other varieties of generate-and-test are willy-nilly. The soliloquy that accompanies the errors committed by the lowliest Skinnerian creature might be “Well, I mustn’t do that again!” and the hardest lesson for any agent to learn, apparently, is how to learn from its own mistakes. In order to learn from them, one has to be able to contemplate them, and this is no small matter. Life rushes on, and unless one has developed positive strategies for recording one’s tracks, the task known in AI as credit assignment (also known, of course, as “blame assignment”) is insoluble.


    `` -- Dennett, Daniel C.. Darwin's Dangerous Idea: Evolution and the Meaning of Life (p. 373). Simon & Schuster. Kindle Edition. ``
}


% beliefs

\newcommand{\qBelief}{``
Nicholas Humphrey (1976, 1983, 1986) has argued that there must be a genetic predisposition for adopting the intentional stance, and Alan Leslie (1992) and others have developed evidence for this, in the form of what he calls a “theory of mind module” designed to generate second-order beliefs (beliefs about the beliefs and other mental states of others). Some autistic children seem to be well described as suffering from the disabling of this module, for which they can occasionally make interesting compensatory adjustments. (For an overview, see Baron-Cohen 1995.) So the words (and hence memes) that take up residence in a brain, like so many earlier design novelties we have considered, enhance and shape preexisting structures, rather than generating entirely new architectures (see Sperber [in press] for a Darwinian overview of this exaptation of genetically provided functions by culturally transmitted

Dennett, Daniel C.. Darwin's Dangerous Idea: Evolution and the Meaning of Life (p. 379). Simon & Schuster. Kindle Edition.
``}


